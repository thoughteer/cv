\documentclass{tidycv}

\input{language}

\cvsetname{\cvenglish{Iskander Sitdikov}\cvrussian{Искандер Ситдиков}}
\cvsetmajor{
    \cvenglish{
        A researcher, an applied mathematician,\\
        a software engineer, and just a nice guy
    }
    \cvrussian{
        Исследователь, прикладной математик,\\
        разработчик ПО и просто хороший человек
    }
}
\cvsetaddress{
    \cvenglish{
        ul.~Leninskie~gory~1\\
        g.~Moskva\\
        119234\\
        Russian~Federation
    }
    \cvrussian{
        ул.~Ленинские~горы~1\\
        г.~Москва\\
        119234\\
        Россия
    }
}
\cvsetphone{+7~963~9667818}
\cvsetemail{thoughteer@gmail.com}
\cvsethomepage{https://bitbucket.org/thoughteer}

\begin{document}
    \cvheader
    \cvfootnote[
        \cvenglish{
            the Faculty of Computational Mathematics and Cybernetics of
            Lomonosov Moscow State University
        }
        \cvrussian{
            Факультет вычислительной математики и кибернетики МГУ имени
            М.\,В.~Ломоносова
        }
    ]{cmcmsu}
    \begin{cvsection}{\cvenglish{Education}\cvrussian{Образование}}
        \cventry{\cvenglish{since}\cvrussian{с} 2013}{
            \textbf{\cvenglish{Ph.\,D.\ degree}\cvrussian{Аспирантура}},
            \cvenglish{CMC MSU}\cvrussian{ВМК МГУ}\cvfootnote{cmcmsu},
            \cvenglish{Moscow, Russia}\cvrussian{Москва, Россия}

            \begin{cvdescription}
                \cvitem{\cvenglish{affiliation}\cvrussian{лаборатория}}{
                    \cvenglish{
                        The Laboratory of Mathematical Methods of Image
                        Processing
                    }
                    \cvrussian{
                        Лаборатория математических методов обработки\\
                        изображений и компьютерного моделирования
                    }
                }
                \cvitem{\cvenglish{supervisor}\cvrussian{руководитель}}{
                    \cvenglish{Prof.~Andrey~S.~Krylov}
                    \cvrussian{проф.,~д.\,ф.-м.\,н.~А.\,С.~Крылов}
                }
            \end{cvdescription}
        }
        \cventry{2008--2013}{
            \textbf{%
                \cvenglish{Specialist (M.\,S.) degree}%
                \cvrussian{Высшее образование}%
            },
            \cvenglish{CMC MSU}\cvrussian{ВМК МГУ}\cvfootnote{cmcmsu},
            \cvenglish{Moscow, Russia}\cvrussian{Москва, Россия}

            \begin{cvdescription}
                \cvitem{\cvenglish{field of study}\cvrussian{специальность}}{
                    \cvenglish{Applied mathematics and computer science}
                    \cvrussian{Прикладная математика и информатика}
                }
                \cvitem{\cvenglish{specialization}\cvrussian{специализация}}{
                    \cvenglish{Mathematical physics}
                    \cvrussian{Математическая физика}
                }
                \cvitem{\cvenglish{affiliation}\cvrussian{лаборатория}}{
                    \cvenglish{
                        The Laboratory of Mathematical Methods of Image
                        Processing
                    }
                    \cvrussian{
                        Лаборатория математических методов обработки\\
                        изображений и компьютерного моделирования
                    }
                }
                \cvitem{\cvenglish{supervisor}\cvrussian{руководитель}}{
                    \cvenglish{Prof.~Andrey~S.~Krylov}
                    \cvrussian{проф.,~д.\,ф.-м.\,н.~А.\,С.~Крылов}
                }
                \cvitem{\cvenglish{GPA}\cvrussian{средний балл}}{
                    5.0 \cvenglish{of}\cvrussian{из} 5
                }
                \cvitem{\cvenglish{thesis}\cvrussian{дипломная работа}}{
                    \cvenglish{
                        Development of combined image enhancement methods
                    }
                    \cvrussian{
                        Построение комбинированных методов повышения\\
                        качества изображений
                    }
                }
            \end{cvdescription}
        }
        \cventry{1998--2008}{
            \textbf{%
                \cvenglish{Secondary (complete) general education}%
                \cvrussian{Среднее (полное) общее образование}%
            },
            \cvenglish{Gymnasium~\textnumero26, Naberezhnye Chelny, Russia}
            \cvrussian{Гимназия~\textnumero26, Набережные Челны, Россия}

            \begin{cvdescription}
                \cvitem{\cvenglish{focus}\cvrussian{профиль}}{
                    \cvenglish{Mathematics and physics}
                    \cvrussian{Математика и физика}
                }
                \cvitem{\cvenglish{GPA}\cvrussian{средний балл}}{
                    5.0 \cvenglish{of}\cvrussian{из} 5
                }
            \end{cvdescription}
        }
    \end{cvsection}
    \begin{cvsection}{\cvenglish{Skills}\cvrussian{Навыки}}
        \cventry{\cvenglish{operating systems}\cvrussian{операционные системы}}{
            Debian (\cvenglish{over}\cvrussian{свыше} \cvyearssince{2010}
                \cvenglish{years}\cvrussian{лет}),
            Microsoft Windows (\cvenglish{over}\cvrussian{свыше}
                \cvyearssince{2006} \cvenglish{years}\cvrussian{лет})
        }
        \cventry{%
            \cvenglish{programming languages}%
            \cvrussian{компьютерные языки}%
        }{
            C\,/\,C++, Python, Matlab, Bash, JavaScript, Factor, SQL
        }
        \cventry{\cvenglish{technologies}\cvrussian{технологии}}{
            CUDA, OpenGL, DICOM, \LaTeX, HTML, CSS
        }
        \cventry{\cvenglish{tools}\cvrussian{инструменты}}{
            Vim, Git, Make, GCC, Visual Studio
        }
    \end{cvsection}
    \begin{cvsection}{\cvenglish{Spoken Languages}\cvrussian{Языки}}
        \cventry{\cvenglish{English}\cvrussian{английский}}{
            \cvenglish{Advanced, fluent}\cvrussian{Продвинутый, разговорный}
        }
        \cventry{\cvenglish{Russian}\cvrussian{русский}}{
            \cvenglish{Native}\cvrussian{Родной}
        }
    \end{cvsection}
    \begin{cvsection}{\cvenglish{Work Experience}\cvrussian{Опыт работы}}
        \cventry{10/2014--12/2014}{
            \textbf{Samsung},
            \cvenglish{Moscow, Russia}\cvrussian{Москва, Россия}

            \cvenglish{
                Worked on an image super-resolution project as a software
                engineer, implementing various motion estimation algorithms.
            }
            \cvrussian{
                Участвовал в проекте по супер-разрешению изображений в качестве
                разработчика, реализовывал различные алгоритмы оценки движения
            }
        }
        \cventry{07/2014--08/2014}{
            \textbf{%
                \cvenglish{The Space Research Institute}%
                \cvrussian{Институт космических исследований РАН}%
            },
            \cvenglish{Moscow, Russia}\cvrussian{Москва, Россия}

            \cvenglish{Engaged in research on satellite auroral imaging:}
            \cvrussian{
                Занимался исследованиями в области обработки спутниковых
                авроральных изображений:
            }
            \begin{cvlist}
                \cvitem{
                    \cvenglish{
                        developing efficient numerical methods for large-scale
                        ground and atmospheric scattering compensation
                    }
                    \cvrussian{
                        разрабатывал эффективные численные методы
                        крупномасштабной компенсации эффектов земного и
                        атмосферного рассеяния
                    }
                }
                \cvitem{
                    \cvenglish{
                        prototyping them in Matlab and implementing in Python
                    }
                    \cvrussian{
                        создавал прототип в Matlab и реализацию на Python
                    }
                }
            \end{cvlist}
        }
        \cventry{07/2011--09/2011 \par 01/2012--02/2012 \par 07/2012--10/2012}{
            \textbf{\cvenglish{The }Advanced Digital Sciences Center},
            \cvenglish{Singapore}\cvrussian{Сингапур}

            \cvenglish{
                Involved in a number of projects as a junior research assistant:
            }
            \cvrussian{
                Был задействован в ряде проектов в качестве младшего научного
                сотрудника:
            }
            \begin{cvlist}
                \cvitem{
                    \cvenglish{real-time stereo matching on a GPU}
                    \cvrussian{
                        стерео-реконструкция в реальном времени на графическом
                        процессоре
                    }
                }
                \cvitem{
                    \cvenglish{Kinect depth map enhancement on a GPU}
                    \cvrussian{
                        повышение качества карт глубины Kinect на графическом
                        процессоре
                    }
                }
                \cvitem{
                    \cvenglish{edge-aware recursive image filtering}
                    \cvrussian{адаптивная рекурсивная фильтрация изображений}
                }
            \end{cvlist}
            \begin{cvdescription}
                \cvitem{\cvenglish{supervisors}\cvrussian{руководители}}{
                    Dr.~Dongbo~Min, Dr.~Kyle~Rupnow
                }
            \end{cvdescription}
        }
    \end{cvsection}
    \begin{cvsection}{\cvenglish{Publications}\cvrussian{Публикации}}
        \cvbibliography
    \end{cvsection}
    \begin{cvsection}{\cvenglish{References}\cvrussian{Рекомендации}}
        \cvreference{%
            \cvenglish{Prof.~Andrey~S.~Krylov}%
            \cvrussian{проф.~А.\,С.~Крылов}%
        }{kryl@cs.msu.ru}{
            \cvenglish{
                Head of the Laboratory of Mathematical Methods of Image
                Processing at CMC MSU\cvfootnote{cmcmsu} (Moscow, Russia).
            }
            \cvrussian{
                Глава лаборатории математических методов обработки изображений и
                компьютерного моделирования ВМК МГУ\cvfootnote{cmcmsu} (Москва,
                Россия).
            }
        }
        \cvreference{Dr.\ Dongbo Min}{dongbo@adsc.com.sg}{
            \cvenglish{
                A former research scientist at the Advanced Digital Sciences
                Center (Singapore).
            }
            \cvrussian{
                Бывший научный сотрудник Advanced Digital Sciences Center
                (Сингапур).
            }
        }
        \cvreference{Dr.\ Kyle Rupnow}{k.rupnow@adsc.com.sg}{
            \cvenglish{
                A research scientist at the Advanced Digital Sciences Center
                (Singapore).
            }
            \cvrussian{
                Научный сотрудник Advanced Digital Sciences Center (Сингапур).
            }
        }
    \end{cvsection}
\end{document}
